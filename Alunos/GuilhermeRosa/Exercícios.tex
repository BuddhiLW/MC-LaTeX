

\documentclass[12pt]{abntex2}
\usepackage{amsmath}
\usepackage{movie15}
\usepackage{graphics,graphicx}
\usepackage{inputenc}


% O tamanho do parágrafo é dado por:
\setlength{\parindent}{1.3cm}

% Controle do espaçamento entre um parágrafo e outro:
\setlength{\parskip}{0.2cm}  % tente também \onelineskip



\setlength\parindent{0pt} %% Do not touch this

%% -----------------------------
%% TITLE
%% -----------------------------
\title{Lista 2} %% Assignment Title

\author{Guilherme R. R. Rosa\\ %% Student name
  Mini-Curso {\LaTeX} \\ %% Code and course name
  \textsc{Universidade de São Paulo}
}

\date{\today} %% Change "\today" by another date manually
%% -----------------------------
%% -----------------------------

%% %%%%%%%%%%%%%%%%%%%%%%%%%
\begin{document}
%% %%%%%%%%%%%%%%%%%%%%%%%%%
\maketitle

% --------------------------
% Start here
% --------------------------

% %%%%%%%%%%%%%%%%%%%


	Problema 1 \\
	2.a
	
	\begin{equation}
	\label{eq:n11}
	\autoref{eq:n11}
	\begin{cases}
	\frac{\mathrm{d}x}{\mathrm{d}t} = \sigma(y-x) \\
	\frac{\mathrm{d}y}{\mathrm{d}t} = x(\rho-z) - y \\
	\frac{\mathrm{d}z}{\mathrm{d}t} = xy - \beta z
	\end{cases}
	\end{equation}

	
%%%%%%%%%%%%%%%%%%%%%%%%%%%%%%%%%%%%%%%%%%%%%%%%%%%%%%%%%%%%%%
%%%%%%%%%%%%%%%%%%%%%%%%%%%%%%%%%%%%%%%%%%%%%%%%%%%%%%%%%%%%%%	
	1.a
	
	\begin{equation}
	\label{eq:n21}
	\autoref{eq:n21}
	\begin{cases}
	F(k)= \frac{1}{2\pi}\int_{-\infty}^\infty s(x) e^{-ikx} \mathrm{d}x
	\end{cases}
	\end{equation}
	\\ \\ \\ \\
	\\ \\ \\ \\
	\\ \\ \\ \\
%%%%%%%%%%%%%%%%%%%%%%%%%%%%%%%%%%%%%%%%%%%%%%%%%%%%%%%%%%%%%%%
%%%%%%%%%%%%%%%%%%%%%%%%%%%%%%%%%%%%%%%%%%%%%%%%%%%%%%%%%%%%%%%
     Problema 2 \\
     
     \begin{table}[htb]
     	\IBGEtab{%
     		\caption{Casos de Covid-19 na região.}%
     		\label{tab:ibge}
     	}{%
     		\begin{tabular}{cccc}
     			\toprule
     			Cidade & Casos (Suspeitos) & Casos(Confirmados) & Mortes  \\
     			\midrule \midrule
     			Holambra & 300                       & 15           & 3 \\
     			\midrule
     			Brotas   & 245                       & 9            & 1 \\
     			\midrule
     			Socorro  & 138                       & 2            & 0 \\
     			\bottomrule
     		\end{tabular}%
     	}{%
     		\fonte{?}%
     	}
     \end{table} 

 \begin{table}[htb]
 	\begin{center}
 		\ABNTEXfontereduzida
 		\caption{\label{tab:formal} Casos de Covid-19 na região}
 		\begin{tabular}{m{2.6cm}|m{4.0cm}|m{2.25cm}|m{3.40cm}}
 			\hline
 			{Cidade} & {Casos (Suspeitos)} & {Casos (Confirmados)} & {Mortes}\\
 			\hline
 			Holambra & {300} & {15} & {3} \\
 			\hline
 			Brotas   & {245}  & {9}  & {1} \\
 			\hline
 			Socorro  & {138}   & {2} & {0} \\
 				\hline
 			\end{tabular}
 			\legend{Fonte:?}
 		\end{center}
 	\end{table}
%%%%%%%%%%%%%%%%%%%%%%%%%%%%%%%%%%%%%%%%%%%%%%%%%%%%%%%%%%%%%%%
%%%%%%%%%%%%%%%%%%%%%%%%%%%%%%%%%%%%%%%%%%%%%%%%%%%%%%%%%%%%%%%
\clearpage


Problema 3

\begin{figure}[htb]
	\begin{center}
		\caption{\label{fig:logo}{\LaTeX} Logo}
		\includegraphics[width=0.85 \textwidth, height=0.65 \textheight]{logo.png}
		\legend{Fonte: https://student.uis.no/library/classes/bachelor/latex/}
	\end{center}
\end{figure}
 

\end{document}

%%% Local Variables:
%%% mode: latex
%%% TeX-master: t
%%% End:
