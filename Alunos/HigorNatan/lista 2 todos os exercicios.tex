\documentclass[a4paper, 12pt]{article}
\usepackage{times}
\usepackage{array}
\usepackage{animate}
\usepackage{amsmath}
\usepackage{multimedia}
\usepackage{graphicx}
\usepackage{esint}
\usepackage[brazilian]{babel}
\usepackage[utf8]{inputenc} %escreva o documento diretamente com os acentos
\usepackage[T1]{fontenc}
\usepackage{booktabs} %tabela boa
\usepackage{caption}
\usepackage{float}
\begin{document}
	\centering EXERCÍCIO 1 A

	\begin{equation}
		\begin{cases}
		\frac{\mathrm{d}x}{\mathrm{d}t} = \sigma(y-x) \\
		\frac{\mathrm{d}y}{\mathrm{d}t} = x(-z)-y \\
		\frac{\mathrm{d}z}{\mathrm{d}t} = xy - \beta z \\
		\end{cases}
	\end{equation}

	\centering EXERCÍCIO 1 B
	\begin{equation}
	\ F(k) = \frac{\mathrm{1}}{\mathrm{2\pi}} \int_{{-\infty}}^{{\infty}} s(x)e^{-ikt} dx
	\end{equation}
	\centering EXERCÍCIO 2
	\begin{table}[!h]
		\centering
		\caption{Índice de Desenvolvimento Humano: Top 5}
		\begin{tabular}{ccc}
		\toprule
			Posição & País & IDH \\
		\midrule
			1 & Noruega & 0.955 \\
			2 & Austrália & 0.938 \\
			3 & EUA & 0.937 \\
			4 & Holanda & 0.921 \\
			5 & Alemanha & 0.920 \\
		\bottomrule
		\end{tabular}
	\vspace{0,1cm}
		\caption*{Fonte: CBF, 2011}
	\end{table}

	\begin{table}[h]
		\centering
		\caption{ Índice de Desenvolvimento Humano: Top 5}
		\vspace{0.5cm}
		\begin{tabular}{r|lr}

			Posição  & País & IDH \\ % Note a separação de col. e a quebra de linhas
			\hline
			\vspace{0.1cm}                               % para uma linha horizontal
			1 & Noruega        &  0.955 \\
			2 & Austrália      &  0.938 \\
			3 & EUA            & 0.937 \\
			4 & Holanda        & 0.921 \\
			5 & Alemanha       & 0.920 \\           % não é preciso quebrar a última linha

		\end{tabular}
	\vspace{0,1cm}
	\caption*{Fonte: UEFA, 2009}

\end{table}

\centering EXERCÍCIO 3
\begin{figure}[H]
	\centering
 % \includegraphics[width = 20.4cm , height = 21.5cm ]{logo}
	\includegraphics[width=1.2\linewidth, height= 1.2 \linewidth]{logo}
	\caption{Logo de Lorena}
\end{figure}



\end{document}
%%% Local Variables:
%%% mode: latex
%%% TeX-master: t
%%% End:
