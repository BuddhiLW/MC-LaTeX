\documentclass[		
oneside,		
a4paper,						
12pt,
brazil]{abntex2}

\usepackage{babel}
\usepackage[utf8]{inputenc}
\usepackage{tcolorbox}
\usepackage{array}
\usepackage{tikz}

\begin{document}


\textbf{Movimento retilíneo uniforme utilizando o colchão de ar linear Hentsche.}\\

Cada intervalo representa o tempo, com suas devidas variações, gasto pelo móvel entre os sensores do aparelho, que distam 10 cm um do outro.\\

\begin{center}
 \textbf{Tabela IBGE}
\end{center}

\begin{table}[h]
	
\IBGEtab {
	\caption{Intervalos de tempo no espaço de 10cm}
	 }
{

\begin{tabular} {cccc}
	\toprule
	1º intervalo & 2º intervalo & 3º intervalo & 4º intervalo\\
	\midrule \midrule
	0,697514s & 0,696571s & 0,692141s & 0,686590\\
	\hline
	0,699051s & 0,685728s & 0,691006s & 0,694233s\\
	\hline
	0,704789s & 0,710141s & 0,710141s & 0,708058s\\
	\hline
	0,690504s & 0,688118s & 0,688118s & 0,692839s\\
	\hline
	0,682013s & 0,687569s & 0,687569s & 0,683203s\\
	\bottomrule
		
\end{tabular}
}
{\fonte{dados experimentais.} }
\end{table}

\begin{center}
	\textbf{Tabela genérica}
\end{center}


\begin{center}
\begin{tcolorbox}
	[title=Intervalos de tempo no espaço de 10cm,title filled,center title,hbox]
	
\begin{tabular}{c|c|c|c}

	
	1º intervalo & 2º intervalo & 3º intervalo & 4º intervalo\\
	\hline\hline
	0,697514s & 0,696571s & 0,692141s & 0,686590\\

	0,699051s & 0,685728s & 0,691006s & 0,694233s\\

	0,704789s & 0,710141s & 0,710141s & 0,708058s\\

	0,690504s & 0,688118s & 0,688118s & 0,692839s\\

	0,682013s & 0,687569s & 0,687569s & 0,683203s\\

\end{tabular}
\end{tcolorbox}
\end{center}

\end{document}