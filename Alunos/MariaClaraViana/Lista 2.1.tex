\documentclass[		
oneside,		
a4paper,						
12pt,
brazil]{article}

\usepackage{esint}
\usepackage{babel}
\usepackage{graphicx}
\graphicspath{{/Users/Maria Clara/Desktop/Figuras/}}
\usepackage[utf8]{inputenc}	

\begin{document}
		
	1.(a)Forma complexa da Equação de Fourier
	\\
	\\
\begin{math}
	$$
	F(k)=\frac{1}{2\pi}\int_{-\infty}^{\infty}{s(x)e^{ikx}dx}
	$$
	\\
	\\
\end{math}

	2.(a)Sistema de equações em que se derivam os atratores de Lorenz
	\\
	\\
\begin{math}
	\left\{ \begin{array}{ll}
	\frac{dx}{dt} & \textrm{$\sigma (x-y)$}\\
	\frac{dy}{dt} & \textrm{$x(\rho -z)-y$}\\
	\frac{dz}{dt} & \textrm{$xy-\beta z$}
	\end{array} \right.
\end{math}

\begin{figure}[h]
	\centering
	\includegraphics[width=10cm]{lorenz}
	\caption{Pêndulo de Lorenz}
\end{figure}
	
	
\end{document}