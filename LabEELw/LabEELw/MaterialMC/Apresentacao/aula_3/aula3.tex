\documentclass{beamer}
% \documentclass{abntex}

% Pacotes
\usepackage[utf8]{inputenc}

%% Configurações do Template, Imagem de Fundo[../template7.jpg], e Tema de apresentação[Madrid]
\usebackgroundtemplate{
  \includegraphics[width=\paperwidth,
  height=\paperheight]{../BackGround4.jpg}
}
\usetheme{Madrid}

%% Mudança de dimenções da apresentação
\mode<presentation>


%% Produção da Capa de Aprensentação
\title[Relatórios]{\Huge{Criando Pacotes e Comandos}}
\author[Branquinho]{Pedro Gomes Branquinho \\
  \text{\scriptsize{pedro.branquinho@usp.br}}}
\date[ABNTeX]{\scriptsize{Mini-curso de \LaTeX \\ Universidade de São Paulo - DEMAR}}


%% Início do documento
\begin{document}

%% Para a Capa, usaremos uma imagem diferente, com o brasão da USP, e logomarca.
{\usebackgroundtemplate{\includegraphics[width=\paperwidth]{../Imagens/TP.jpg}}
  \begin{frame}
    \titlepage
  \end{frame}
}

%% Notas de versões anteriores
% \begin{frame}
%   \frametitle{Apesentação em \LaTeX}
%   \tableofcontents[pausecontents]
% \end{frame}


%% Início da Apresentacão, o que esperar
\begin{frame}
  \section{Motivações}
  \frametitle{Motivações}

  \begin{enumerate}
  \item[1] Aumentar produtividade
  \item[2] Entender a estrutura de pacotes
  \item[3] Criar seus próprios templates
  \end{enumerate}

\end{frame}

\begin{frame}[fragile]
  \section{Criando Comandos}
  \frametitle{Criando Comandos} %newcommand}

  \begin{center}

  \setbeamercolor{block title}{fg=white,bg=blue!75!black}
  \begin{block}{Novo comando}
    Usamos o comando
    \verb+\newcommand{name}[num][default]{definition}+ para criar um
    novo comando.

  \end{block}

  \vspace{0.2cm}

  \begin{itemize}
  \item Dá-se o nome do comando em``name'' i.e.,
  ``\verb+\nome_do_comando+'';

  \vspace{0.2cm}

   \item O número de argumentos que o novo
     comando vai receber em ``num''.

     \vspace{0.2cm}

   \item Por fim, a área de definição é onde se programa os comportamentos do comando.

  \end{itemize}
\end{center}

\end{frame}

\begin{frame}

  \frametitle{Exemplo 1}



\end{frame}


\end{document}

%%% Local Variables:
%%% mode: latex
%%% TeX-master: t
%%% End:

%%% Local Variables:
%%% mode: latex
%%% TeX-master: t
%%% End:
