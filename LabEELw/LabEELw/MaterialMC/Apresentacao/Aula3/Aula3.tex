\documentclass{beamer}
% \documentclass{abntex}

% Pacotes
\usepackage[utf8]{inputenc}

%\usepackage{fontspec}
%\usepackage{xunicode}
%\usepackage{xltxtra}
% \usepackage{wasysym}
% \usepackage{multirow}
% \usepackage{tikz}
% \usepackage{alltt}
% \usepackage{ifthen}
% \newboolean{slides}
% \setboolean{slides}{true}
\setbeamertemplate{footline}[frame number]



%% Configurações do Template, Imagem de Fundo[../template7.jpg], e Tema de apresentação[Madrid]
\usebackgroundtemplate{
  \includegraphics[width=\paperwidth,
  height=\paperheight]{../BackGround4.jpg}
}
\usetheme{Madrid}

%% Mudança de dimenções da apresentação
\mode<presentation>


%% Produção da Capa de Aprensentação
\title[Relatórios]{\Huge{Criando Pacotes e Comandos}}
\author[Branquinho]{Pedro Gomes Branquinho \\
  \text{\scriptsize{pedro.branquinho@usp.br}}}
\date[ABNTeX]{\scriptsize{Mini-curso de \LaTeX \\ Universidade de São Paulo - DEMAR}}


%% Início do documento
\begin{document}

%% Para a Capa, usaremos uma imagem diferente, com o brasão da USP, e logomarca.
{\usebackgroundtemplate{\includegraphics[width=\paperwidth]{../Imagens/TP.jpg}}
  \begin{frame}
    \titlepage
  \end{frame}
}

%% Notas de versões anteriores
% \begin{frame}
%   \frametitle{Apesentação em \LaTeX}
%   \tableofcontents[pausecontents]
% \end{frame}
\usenavigationsymbolstemplate{}



  % \AtBeginSection[]{
  %   \addtocounter{framenumber}{-1}
  %   \begin{frame}<beamer>
  %     \frametitle{Outline}
  %     \begin{small}
  %       % \tableofcontents[hidesubsections,currentsection] %,currentsubsection]
  %       \tableofcontents[shadesubsections,currentsection] %,currentsubsection]
  %     \end{small}
  %   \end{frame}
  % }}{}


%% Início da Apresentacão, o que esperar
\section{Motivações}
\begin{frame}
  \frametitle{Motivações}

  \begin{enumerate}
  \item Aumentar produtividade
  \item Entender a estrutura de pacotes
  \item Criar seus próprios templates
  \end{enumerate}

\end{frame}

\begin{frame}[fragile]

  \section{Criando Comandos}
  \frametitle{ \verbatim+\newcommand+ }
  \pause
  %\setbeamercolor{block title}{fg=white,bg=blue!75!black}
  \begin{block}{ABN\TeX}
    O pacote abnTeX se trata da construção de comandos de formatações
    dentro das normas a ABNT NBR 14724:2011 e a ABNT NBR 6024:2012, as
    quais englobarem os requisitos das demais normas ABNT de produção
    textual. Utilizaremos largamente os \alert{``Modelos Canônicos''} da classe.
  \end{block}

  \pause
  %\setbeamercolor{block title}{fg=white,bg=red!75!black}
  \begin{block}{Modelos Canônicos}
    São documentos os quais seguem estritamente as normas ABNT. Porém,
    não necessariamente as especificações de uma intituição. As
    instituições brasileiras adotam particularidades, com pequenas
    variações, em relação aos modelos canônicos.
  \end{block}

\end{frame}


\begin{frame}

  \frametitle{Usando o pacote abnTeX}
  \pause

  No preâmbulo do documento, carregue os pacotes presentes
  no modelo canônico de relatórios técnicos. E, a classe do documento
  como ``abntex2'' - as opções, 12pt, openright, etc. são parâmetros configuráveis.

  \pause
  \begin{center}
    \includegraphics<-3>[width=8cm,height=7cm]{../Imagens/A2I1.png}
  \end{center}

\end{frame}

\begin{frame}
  \section{Funcionalidades}
  \frametitle{Sumário, Indíces}

  \begin{itemize}
  \item Comando para produção de Sumário
  \end{itemize}

  % \pause

  \begin{center}
    \includegraphics[scale=0.6]{../Imagens/A2I2.png}
    % \includegraphics[width=5.5cm,height=5cm]{../Imagens/A2I22.png}
  \end{center}
\end{frame}




\begin{frame}
  % \section{Funcionalidades}
  \frametitle{Sumário, Indíces}

  \begin{itemize}
  \item Auto-produção dos Índices, Indexação
  \end{itemize}

  % \pause

  \begin{center}
    \includegraphics[width=5.5cm,height=5cm]{../Imagens/A2I21.png}
    \includegraphics[width=5.5cm,height=5cm]{../Imagens/A2I22.png}
  \end{center}
\end{frame}


\begin{frame}
  % \section{Funcionalidades}
  \frametitle{Sumário, Indíces}

  \begin{itemize}
  \item Auto-produção dos Índices, Indexação
  \end{itemize}

  \begin{center}
    \includegraphics[width=5.5cm,height=5cm]{../Imagens/A2I31.png}
    \includegraphics[width=5.5cm,height=5cm]{../Imagens/A2I32.png}
  \end{center}
\end{frame}


% \pause

% \begin{center}
%   % \includegraphics[width=4cm,height=3.5cm]{../Imagens/A2I31.png}
%    % \includegraphics[width=4cm,height=3.5cm]{../Imagens/A2I32.png}
% \end{center}

\begin{frame}
  % \section{Funcionalidades}
  \frametitle{Sumário}

  \begin{itemize}
  \item Formatação textual lógica-sequencial
  \end{itemize}

  \begin{center}
    \includegraphics[width=12cm,height=7cm]{../Imagens/A2I41.png}
    % \includegraphics[width=5.5cm,height=5cm]{../Imagens/A2I42.png}
  \end{center}

\end{frame}

\begin{frame}
  % \section{Funcionalidades}
  \frametitle{Sumário}

  \begin{itemize}
  \item Auto-produção dos Sumários
  \end{itemize}

  \begin{center}
    % \includegraphics[width=5.5cm,height=5cm]{../Imagens/A2I41.png}
    \includegraphics[width=7cm,height=5cm]{../Imagens/A2I42.png}
  \end{center}

\end{frame}

% \begin{frame}
% % \section{Funcionalidades}
% \frametitle{Sumário}

% \begin{itemize}
% \item Auto-produção dos Sumários
% \end{itemize}

% \begin{center}
% % \includegraphics[width=5.5cm,height=5cm]{../Imagens/A2I41.png}
% \includegraphics[width=7cm,height=5cm]{../Imagens/A2I42.png}
% \end{center}

% \end{frame}


\begin{frame}
  % \section{Funcionalidades}
  \frametitle{Contra-exemplo}

  \begin{itemize}
  \item Eventual adição de itens ao documento
  \end{itemize}

  \begin{center}
    \includegraphics[scale=0.185]{../Imagens/A2I51.png}
  \end{center}

\end{frame}


\begin{frame}
  % \section{Funcionalidades}
  \frametitle{Contra-exemplo}

  \begin{itemize}
  \item Comportamento esperado, índice-textual:
  \end{itemize}

  \begin{center}
    \includegraphics[scale=0.25]{../Imagens/A2I53.png}
  \end{center}

\end{frame}


\begin{frame}
  % \section{Funcionalidades}
  \frametitle{Contra-exemplo}

  \begin{itemize}
  \item Comportamento esperado, sumário:
  \end{itemize}

  \begin{center}
    \includegraphics[scale=0.3]{../Imagens/A2I52.png}
  \end{center}

\end{frame}
% \begin{frame}
% \item<4-> Formatação de Figuras - Imagens e Gráficos
% \end{frame}

\begin{frame}
  \section{Tabelas - Modelo IBGE}
  \frametitle{Tabelas - Modelo IBGE}

  \begin{itemize}
  \item Ambiente Tabela || || ||  || Resultado
  \end{itemize}

  \begin{center}
    \includegraphics[scale=0.28]{../Imagens/A2I61.png}
    \includegraphics[scale=0.24]{../Imagens/A2I62.png}
  \end{center}

  \pause

  Nota: existe exemplos dessas formatações nos arquivos do
  repositório. Basta reutilizá-los para tabelas, imagens, e fórmulas.

\end{frame}

\begin{frame}
  \section{Fórmulas Matemáticas}
  \frametitle{Fórmulas Matemáticas}

  \begin{itemize}
  \item Ambiente de Fórmula || || || Resultado
  \end{itemize}
  % \~\~ \\
  % \~\~ \\
  \begin{center}
    \includegraphics[scale=0.30]{../Imagens/A2I71.png}
    % \~\~ \\
    % \~\~ \\
    \includegraphics[scale=0.40]{../Imagens/A2I72.png}
  \end{center}
\end{frame}



\begin{frame}
  \section{Formatação de  Imagens}
  \frametitle{Ambiente de Imagens}
  % \\~~
  Esse é o código usado para formatar uma imagem da apresentação:
  \begin{center}
    \includegraphics[scale=0.30]{../Imagens/A2I91.png}
  \end{center}
  % \\~~
  Resultado:
  \begin{center}
    \includegraphics[scale=0.23]{../Imagens/A2I92.png}
  \end{center}
\end{frame}


\begin{frame}
  \section{Formatação de  Imagens}
  \frametitle{Ambiente de Imagens}
  %\\~~
  Esse é o código usado para formatar uma imagem da apresentação:
  \begin{center}
    \includegraphics[scale=0.30]{../Imagens/A2I91.png}
  \end{center}
  %\\~~
  Resultado:
  \begin{center}
    \includegraphics[scale=0.23]{../Imagens/A2I92.png}
  \end{center}
\end{frame}

\begin{frame}
  \section{Formatação de Referências}
  \frametitle{Arquivo ``.bib''}

  \begin{itemize}
  \item O arquivo .bib é um banco de dados estruturado, de onde chama-se
    dados para construir citações e refências bibliográficas.
  \end{itemize}

  %\\~~

  Exemplo do conteúdo do arquivo,
  \begin{center}
    \includegraphics[scale=0.40]{../Imagens/A2I101.png}
  \end{center}
\end{frame}

\begin{frame}
  \frametitle{Referência tipo BibTeX}

  \begin{itemize}
  \item O nome dado a essa estrutura dos dados é do tipo BibTex. E, é
    possível acessá-las prontas na Internet.
  \end{itemize}

  %\\~~

  Pelo Google Scholar,
  \begin{center}
    \includegraphics[scale=0.20]{../Imagens/A2I102.png}
  \end{center}
  %\\~~
  (Clicar no símbolo de aspas: '' - canto inferior esquerdo do artigo)


\end{frame}

\begin{frame}

  \frametitle{Citação formato BibTeX}

  \begin{itemize}
  \item É possível escolher formatações de diversos bancos de
    dados. Utilizaremos a opção BibTex.
  \end{itemize}

  %\\~~

  \begin{center}
    \includegraphics[scale=0.15]{../Imagens/A2I103.png}
    (Clicar em BibTex, canto inferior esquerdo)
  \end{center}


\end{frame}

\begin{frame}

  \frametitle{Citação formato BibTeX}

  \begin{itemize}
  \item Copiar e colar a informação para o seu arquivo de extenção .bib
  \end{itemize}

  %\\~~

  \begin{center}
    \includegraphics[scale=0.30]{../Imagens/A2I104.png}
  \end{center}


\end{frame}


\begin{frame}[fragile]

  \frametitle{Citações - Documento Principal}

  \begin{itemize}
  \item Comando \verb_\cite{AutorAno}_ e seu resultado, na prática.
  \end{itemize}

  %\\~~

  \begin{center}
    Comando: \\
    \includegraphics[scale=0.30]{../Imagens/A2I111.png}

   % \\~~

    Resultado: \\
    \includegraphics[scale=0.40]{../Imagens/A2I112.png}
  \end{center}

\end{frame}

\begin{frame}[fragile]
  \frametitle{Referências Bibliográficas - Documento Principal}
  \begin{itemize}
  \item Comando \verb_\bibliography{Diretório/Nome-Arquivo-Bib}_ e seu resultado, na prática.
  \end{itemize}
  %\\~~
  \begin{center}
    Comando, ao fim do documento:
   % \\~~
    \includegraphics[scale=0.60]{../Imagens/A2I122.png}
  \end{center}

\end{frame}

\begin{frame}[fragile]
  \frametitle{Referências Bibliográficas - Documento Principal}
  \begin{itemize}
  \item Comando \verb_\bibliography{Diretório/Nome-Arquivo-Bib}_ e seu resultado, na prática.
  \end{itemize}
  \begin{center}
    Resultado:
    \includegraphics[scale=0.30]{../Imagens/A2I121.png}
  \end{center}
\end{frame}

% \begin{frame}[fragile]

%   \frametitle{Referências Fórmulas e Tabelas}

%   \begin{itemize}
%   \item Comandos \verb_\label{meuLabel}_  e \verb_\autoref{meuLabel}_
%   \end{itemize}

%   \begin{center}
%     Resultado: \\
%     \includegraphics[scale=0.30]{../Imagens/A2I121.png}
%   \end{center}

% \end{frame}



\end{document}

%%% Local Variables:
%%% mode: latex
%%% TeX-master: t
%%% End:

%%% Local Variables:
%%% mode: latex
%%% TeX-master: t
%%% End:
