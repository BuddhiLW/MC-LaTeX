%%% Local Variables:
%%% mode: latex
%%% TeX-master: t
%%% End:


%%% Template originaly created by Karol Kozioł (mail@karol-koziol.net) and modified for ShareLaTeX use

\documentclass[a4paper,11pt, dvipdfmx]{abntex2}

\usepackage[T1]{fontenc}
\usepackage[utf8]{inputenc}
\usepackage{graphicx}
\usepackage{xcolor}
\usepackage{movie15}

\renewcommand\familydefault{\sfdefault}
\usepackage{tgheros}
% \usepackage{Consolatas}
\usepackage{yfonts}

\usepackage{amsmath,amssymb,amsthm,textcomp}

\setlist[2]{noitemsep} % sets the itemsep and parsep for all level two lists to 0

\setenumerate{noitemsep} % sets no itemsep for enumerate lists only
% \begin{enumerate}[noitemsep] % sets no itemsep for just this list
%   % ...
% \end{enumerate}


\usepackage{multicol}
\usepackage{tikz}

\usepackage[margin=1in]{geometry}
\geometry{left=25mm,right=25mm,%
  bindingoffset=0mm, top=20mm,bottom=20mm}


\linespread{1.1}

\newcommand{\linia}{\rule{\linewidth}{0.5pt}}

% custom theorems if needed
\newtheoremstyle{mytheor}
{1ex}{1ex}{\normalfont}{0pt}{\scshape}{.}{1ex}
{{\thmname{#1 }}{\thmnumber{#2}}{\thmnote{ (#3)}}}

\theoremstyle{mytheor}
\newtheorem{defi}{Definition}

% my own titles
\makeatletter
\renewcommand{\maketitle}{
  \begin{center}
    \vspace{4ex}
    { {\fontsize{40}{40}\selectfont \textfrak{\@title}} }
    \vspace{2ex}
    \\
    \linia\\
    \@author \hfill \@date
    \vspace{6ex}
  \end{center}
}
\makeatother
%%%

% custom footers and headers
\usepackage{fancyhdr}
\pagestyle{fancy}
\lhead{}
\chead{}
\rhead{Entrega até 14/05}
\lfoot{Lista \textnumero{} 1}
\cfoot{}
\rfoot{Página \thepage}
\renewcommand{\headrulewidth}{0.5pt}
\renewcommand{\footrulewidth}{0.5pt}
%

%%% ----------%%%----------%%%----------%%%----------%%%

\usepackage{hyperref}

{%Muda a cor do Sumário, pois são todo links.
  \hypersetup{
    colorlinks=true,
    citecolor= violet,
    linkcolor=black!85,
    filecolor=magenta,
    urlcolor=cyan,
  }

  %%% ----------%%%----------%%%----------%%%----------%%%
  %% Animação
  %\usepackage[loop]{animate}
 % \usepackage{animate}
  %\usepackage{media9}
%  \usepackage{graphicx}


  \begin{document}

  \title{Lis:ta 3}

  \author{\indent  Prof.: Pedro G. Branquinho, EEL-USP. \\
    Orient.: Katia C. G. Candioto.}

  \date{16/05/20 - 24/05/20}

  \maketitle

  \section*{Problema 1}
\textbf{1.(a) Reproduza uma apresentação de dois, ou mais, frames.} O
primeiro frame deve ser sua capa de aprensentação, com seu nome e e-mail
usp, com algum título que lhe agrade. O template de
\href{https://github.com/26-55-87-BuddhiLW/MC-LaTeX/tree/master/Exemplos/ArquivosCurso/ModApresent}{apresentação}
pode lhe ajudar.

\vspace{10mm}

\begin{flushleft}
\textbf{1.(b) No(s) seguinte(s) frame(s)} devem haver,

\begin{itemize}
\item uma caixa, do pacote tboxcolor;
\item um ambiente itemize com uma citação, usando \verb+\cite{}+ nas normas ABNT;
\item uma imagem não distorcida, preferencialmente, dentro de uma caixa.
\end{itemize}

\end{flushleft}
\end{document}

%%% Local Variables:
%%% mode: latex
%%% TeX-master: t
%%% End:
